%%%%%%%%%%%%%%%%%%%%%%%%%%%%%%%%%%%%%%%%%
% Stylish Curriculum Vitae
% LaTeX Template
% Version 1.0 (18/7/12)
%
% This template has been downloaded from:
% http://www.LaTeXTemplates.com
%
% Original author:
% Stefano (http://stefano.italians.nl/)
%
% IMPORTANT: THIS TEMPLATE NEEDS TO BE COMPILED WITH XeLaTeX
%
% License:
% CC BY-NC-SA 3.0 (http://creativecommons.org/licenses/by-nc-sa/3.0/)
%
% The main font used in this template, Adobe Garamond Pro, does not 
% come with Windows by default. You will need to download it in
% order to get an output as in the preview PDF. Otherwise, change this 
% font to one that does come with Windows or comment out the font line 
% to use the default LaTeX font.
%
%%%%%%%%%%%%%%%%%%%%%%%%%%%%%%%%%%%%%%%%%

\documentclass[a4paper, oneside, final]{scrartcl} % Paper options using the scrartcl class
\usepackage{scrpage2} % Provides headers and footers configuration
\usepackage{titlesec} % Allows creating custom \section's
\usepackage{marvosym} % Allows the use of symbols
\usepackage{tabularx,colortbl} % Advanced table configurations
\usepackage{fontspec} % Allows font customization
\usepackage{hyperref}
\usepackage{mdwlist}

\usepackage{fontspec}

\usepackage{polyglossia}
\setmainlanguage{portuges}

\defaultfontfeatures{Mapping=tex-text}
% \setmainfont{Adobe Garamond Pro} % Main document font

\titleformat{\section}{\large\scshape\raggedright}{}{0em}{}[\titlerule] % Section formatting

\pagestyle{scrheadings} % Print the headers and footers on all pages

\addtolength{\voffset}{-0.5in} % Adjust the vertical offset - less whitespace at the top of the page
\addtolength{\textheight}{3cm} % Adjust the text height - less whitespace at the bottom of the page

\newcommand{\gray}{\rowcolor[gray]{.90}} % Custom highlighting for the work experience and education sections

%----------------------------------------------------------------------------------------
% �FOOTER SECTION
%----------------------------------------------------------------------------------------

\renewcommand{\headfont}{\normalfont\rmfamily\scshape} % Font settings for footer

\cofoot{
\addfontfeature{LetterSpace=20.0}\fontsize{12.5}{17}\selectfont % Letter spacing and font size

Travessa Castro Alves {\large\textperiodcentered} Itabuna {\large\textperiodcentered} Brazil 45600-000\\ % Your mailing address
{\Large\Letter} thalles753@gmail.com \ {\Large\Telefon} +55 (73) 8222-8989 % Your email address and phone number
}

%----------------------------------------------------------------------------------------

\begin{document}

\begin{center} % Center everything in the document

%----------------------------------------------------------------------------------------
% �HEADER SECTION
%----------------------------------------------------------------------------------------

{\addfontfeature{LetterSpace=20.0}\fontsize{36}{36}\selectfont\scshape Thalles Santos Silva} % Your name at the top

\vspace{1.5cm} % Extra whitespace after the large name at the top

%----------------------------------------------------------------------------------------
%	OBJECTIVE
%----------------------------------------------------------------------------------------

% USELESS SECTION
%\section{Objective}
%
%A position in the field of computers with special interests in business applications \\ programming, information processing, and management systems.

%----------------------------------------------------------------------------------------
%	EDUCATION
%----------------------------------------------------------------------------------------

\section{Education}

\begin{tabularx}{0.97\linewidth}{>{\raggedleft\scshape}p{2.34cm}X}
    \gray Period & \textbf{July 2009 --- December 2014}\\
    \gray Degree & \textbf{Bachelor in Computer Science}\\
	\gray University & \textbf{The State University of Santa Cruz} \hfill Ilhéus, Brazil\\
	Course Work & Operating Systems; Data Bases; Algorithms; Programming Languages; Computer Architecture; Statistics; Calculus; Parallel Processing\\
& 
\end{tabularx}

\vspace{12pt}

\begin{tabularx}{0.97\linewidth}{>{\raggedleft\scshape}p{2.34cm}X}
	\gray Period & \textbf{July 2013 --- May 2014}\\
	\gray Degree & \textbf{Bachelor in Computer Science}\\
	\gray University & \textbf{Algoma University} \hfill Sault Ste Marie, Canada\\
	Exchange & \textbf{Science Without Borders} - Brazilian Government.
\end{tabularx}

\begin{itemize}

\item \textbf{Theses in Computer Science:} Parallel Processing Using Hybrid Techniques for Applications in Transport of Particles based on Monte Carlo method. This work emphasizes the benefits of implementing parallel scientific applications using NVIDIA CUDA, OpenMP and MPI over common serial applications. Technologies: NVIDIA CUDA, OpenMP e MPI.
   
\item \textbf{EasyKanban (2012)}. Web-based application for project management; Built-in Features: create and manage tasks, create and manage projects, identify bottlenecks in the development process. Technologies: HTML5, CSS3, SQL, JavaScript, PHP, AJAX. \textit{Awarded First Prize in Computer Science Software Engineering Class Contest}.
  
\item \textbf{Application for compressing and encrypting text files} using Huffman algorithm which can reduce the file's size by 60\%. \textit{Awarded First Prize in Computer Science Programming Class Contest}, (2010). 
  
\end{itemize}



%----------------------------------------------------------------------------------------
%	WORK EXPERIENCE
%----------------------------------------------------------------------------------------

\section{Work Experience}

\begin{tabularx}{0.97\linewidth}{>{\raggedleft\scshape}p{2.2cm}X}
\gray Period & \textbf{May 2014 --- November 2014}\\
\gray Employer & \textbf{Great Lakes Forestry Centre } \hfill Sault Ste Marie, Canada\\
\gray Position & \textbf{Researcher Junior}\\
\gray Languages & \textbf{R, Linux Shell Script}\\

\end{tabularx}

\begin{itemize}

\item Led implementation of the R opentraj package for creating and analyzing air trajectory data, which was the main computational tool for a research project on insects' transportation.\\

\item Developed shell scripts for running existing software, such as \textit{Hybrid Single Particle Lagrangian Integrated Trajectory Model} (HYSPLIT), without its GUI interface in order to redirect its results into the R environment.

\end{itemize}

\vspace{12pt}

\begin{tabularx}{0.97\linewidth}{>{\raggedleft\scshape}p{2.2cm}X}
\gray Period & \textbf{March 2012 --- June 2013 (Part Time)}\\
\gray Employer & \textbf{RCS Informática} \hfill Itabuna, Brazil\\
\gray Position & \textbf{Software Developer}\\
\gray Languages & \textbf{Delphi, HTML5, CSS3, JavaScript, PHP, MySQL}\\

\end{tabularx}

\begin{itemize}
\item Developed and tested software, such as Protocol System, Accounting System, and Daily Management System, which led to a percent increase of 5\% in revenue.\\

\item Redesigned the Enterprise's website which led to a decreasing in the site' loading and processing by making asynchronous calls to the server and maximizing the work done on the client side, leading to a 20\% reduction in custumer support calls.\\

\item Wrote SQL scripts for retrieving data from FireBird databases to Delphi applications.\\

\item Debugged and tested existing software plug-ins for user management and report generation.

\end{itemize}

\vspace{12pt}

\begin{tabularx}{0.97\linewidth}{>{\raggedleft\scshape}p{2.2cm}X}
\gray Period & \textbf{July 2010 --- July 2013 (Part Time)}\\
\gray Employer & \textbf{The State University of Santa Cruz} \hfill Ilhéus, Brazil\\
\gray Position & \textbf{Student Researcher}\\
\gray Languages & \textbf{C/C++, Shell Script, NVIDIA CUDA, OpenMP, MPI}\\

\end{tabularx}

\begin{itemize}

\item Student Researcher sponsored by Higher Education Personnel Training Coordination (CAPES).

\item Used the GNU Compiler Collection (GCC) and its directives for developing and optimizing serial programs up to 1.6 times more efficient than common implementations.

\item Developed hybrid parallel programs for particles transport simulation using OpenMP and MPI libraries up to 16 times more efficient than optimized serial implementations.

\item Implemented parallel programs using NVIDIA CUDA up to 2 times more efficient than the versions with OpenMP and MPI.

\end{itemize}

%----------------------------------------------------------------------------------------
%	EXTRA CURRICULAR ACTIVITIES
%----------------------------------------------------------------------------------------
\section{Extracurricular Activities}
\begin{itemize}

\item \textbf{English as a Second Language (ESL)}, Algoma University intensive English Course - (2013) \textit{http://www.algomau.ca/esl/}.

\item \textbf{Introduction to Computer Science Building a Search Engine}, Udacity, \linebreak \textit{www.udacity.com/course/cs101}.

\item \textbf{Introduction to Parallel Programming using NVIDIA C/C++ CUDA}, Udacity, \textit{https://www.udacity.com/course/cs344}.

\item \textbf{C++ for C Programmers}, Coursera and University of California, Santa Cruz, \textit{https://www.coursera.org/course/cplusplus4c}.

\item \textbf{Getting Started with GIS}, ESRI, \textit{http://www.esri.com/training/main}, (ArcGIS 10.0).

\item \textbf{Achieve Languages}, English Course - (2009-2011).

\end{itemize}

%----------------------------------------------------------------------------------------
%	SKILLS
%----------------------------------------------------------------------------------------

\section{Skills}

\begin{tabular}{ @{} >{\bfseries}l p{9cm} l  }
Languages & C; C++; Java; R; HTML; PHP; JavaScript; Python; SQL; CUDA; OpenMP; MPI; \\
Data Bases & MySQL, PostgreSQL, Oracle Enterprise Edition; \\
Technologies & XCode; Eclipse; NetBeans; SVN; Git; \\
Certifications & JavaScript\\
Operating Systems & Linux; MAC OS; Windows;
\end{tabular}

%----------------------------------------------------------------------------------------
%	IDIOMAS
%----------------------------------------------------------------------------------------

\section{Languages}

\begin{itemize} \itemsep1pt \parskip0pt \parsep0pt
  \item English
  \item Portuguese 
\end{itemize}

%----------------------------------------------------------------------------------------
%	INTERESSES
%----------------------------------------------------------------------------------------

\section{Interests}

\begin{flushleft}
High Performance Computing, Linux, Big Data, Programming Languages, Machine Learning, Software Development.
\end{flushleft}

%----------------------------------------------------------------------------------------
%	REFERENCIAS
%----------------------------------------------------------------------------------------

\section{References}

\begin{itemize} \itemsep4pt \parskip0pt \parsep0pt
\item \textbf{Dr Jean-Noel Candau}, Scientist, Great Lakes Forestry Centre, Sault Ste Marie, Ontario, Canada. \textit{Jean-Noel.Candau@NRCan-RNCan.gc.ca}

\item \textbf{Dr Esbel Tomás Valero Orellana}, Teacher, The State University of Santa Cruz, Ilhéus, Bahia, Brazil. \textit{valero.esbel@gmail.com} 

\item \textbf{Mydiã Falcão Freitas}, Software Engineer, RCS Informática, Itabuna, Bahia, Brazil. \textit{mydyfreitas@gmail.com} 
\end{itemize}

\end{center}

\end{document}