
%%%%%%%%%%%%%%%%%%%%%%%%%%%%%%%%%%%%%%%%%
% Stylish Curriculum Vitae
% LaTeX Template
% Version 1.0 (18/7/12)
%
% This template has been downloaded from:
% http://www.LaTeXTemplates.com
%
% Original author:
% Stefano (http://stefano.italians.nl/)
%
% IMPORTANT: THIS TEMPLATE NEEDS TO BE COMPILED WITH XeLaTeX
%
% License:
% CC BY-NC-SA 3.0 (http://creativecommons.org/licenses/by-nc-sa/3.0/)
%
% The main font used in this template, Adobe Garamond Pro, does not 
% come with Windows by default. You will need to download it in
% order to get an output as in the preview PDF. Otherwise, change this 
% font to one that does come with Windows or comment out the font line 
% to use the default LaTeX font.
%
%%%%%%%%%%%%%%%%%%%%%%%%%%%%%%%%%%%%%%%%%

\documentclass[10pt, a4paper, oneside, final]{scrartcl} % Paper options using the scrartcl class
\usepackage{scrlayer-scrpage} % Provides headers and footers configuration
\usepackage{titlesec} % Allows creating custom \section's
\usepackage{marvosym} % Allows the use of symbols
\usepackage{tabularx,colortbl} % Advanced table configurations
\usepackage{fontspec} % Allows font customization
\usepackage{hyperref}
\usepackage{mdwlist}
\usepackage[top=2.6cm,left=1.4cm,right=1.4cm,bottom=1.3cm]{geometry} %margens
\usepackage{fontspec}

\usepackage{polyglossia}
\setmainlanguage{english}

\defaultfontfeatures{Mapping=tex-text}
% \setmainfont{Adobe Garamond Pro} % Main document font

\titleformat{\section}{\large\scshape\raggedright}{}{0em}{}[\titlerule] % Section formatting

%\addtolength{\voffset}{-0.2in} % Adjust the vertical offset - less whitespace at the top of the page
%\addtolength{\textheight}{-0.1in} % Adjust the text height - less whitespace at the bottom of the page

\newcommand{\gray}{\rowcolor[gray]{.90}} % Custom highlighting for the work experience and education sections

%----------------------------------------------------------------------------------------
% HEADER SECTION
%----------------------------------------------------------------------------------------
\lohead{\addfontfeature{LetterSpace=20.0}\fontsize{16}{26}\selectfont\scshape Thalles Santos Silva}

\renewcommand{\headfont}{\normalfont\rmfamily\scshape} % Font settings for footer
\rohead{
	\addfontfeature{LetterSpace=20.0}\fontsize{12}{10}
	Culto à Ciência Street {\large\textperiodcentered} Campinas {\large\textperiodcentered} Brazil 13020-061\\
	{\Large\Letter} thalles753@gmail.com \ {\Large\Telefon} +55 (73) 98222-8989\\
	\url{https://github.com/thalles753/}}
}

\pagestyle{scrheadings} % Print the headers and footers on all pages

\begin{document}

%\begin{center} % Center everything in the document
\pagenumbering{gobble} % remove the page number

%----------------------------------------------------------------------------------------
%	WORK EXPERIENCE
%----------------------------------------------------------------------------------------

\section{Work Experience}

\begin{center}
\begin{tabularx}{1.0\linewidth}{>{\raggedleft\scshape}p{2.2cm}X}
\gray Employer & \textbf{Eldorado Institute of Technology} \hfill Campinas, Brazil - \textbf{May 2015 --- Currently}\\
\gray Position & \textbf{Software Engineer}\\
% \gray Languages & \textbf{C\#, C++, JavaScript; Python }\\
\end{tabularx}
\end{center}

\begin{itemize}\itemsep1.0pt
\item Maintained a constant interaction with project's customers which led to high feed backs on customers satisfaction.

%\item Scrum based projects.

%\item Contributed to some of the most relevant projects in the field of image processing. 

\item Contributor to the internal Machine Learning research group that allowed the company to get new customers. 
\end{itemize}

\begin{center}
\begin{tabularx}{1.0\linewidth}{>{\raggedleft\scshape}p{2.2cm}X}
\gray Employer & \textbf{Great Lakes Forestry Centre } \hfill Sault Ste Marie, Canada - \textbf{May 2014 --- November 2014}\\
\gray Position & \textbf{Researcher Junior}\\
% \gray Languages & \textbf{R, Linux Shell Script}\\
\end{tabularx}
\end{center}

\begin{itemize}\itemsep1.0pt

\item Led implementation of the R opentraj package for creating and analyzing air trajectory data which was the main computational tool for a research project on insects' transportation.

\item Library that encapsulates the core functionalities of the \textit{Hybrid Single Particle Lagrangian Integrated Trajectory Model} (HYSPLIT) software in order to have total access of its results from within the R environment.

\end{itemize}

\begin{center}
\begin{tabularx}{1.0\linewidth}{>{\raggedleft\scshape}p{2.2cm}X}
\gray Employer & \textbf{RCS Informática} \hfill Itabuna, Brazil - \textbf{March 2012 --- June 2013}\\
\gray Position & \textbf{Software Developer (Part Time)}\\
% \gray Languages & \textbf{Delphi, HTML5, CSS3, JavaScript, PHP, MySQL}\\
\end{tabularx}
\end{center}

\begin{itemize}\itemsep1.0pt
\item Contributed to developing Protocol and Daily Management systems which led to an increase in company's revenue.

\item Redesigned the Enterprise's website leading to a significant reduction in customer supporting calls.

% \item Wrote SQL scripts for retrieving data from FireBird databases to Delphi applications.

% \item Debugged existing software plug-ins for user management and report generation.

\end{itemize}

%\begin{center}
%\begin{tabularx}{1.0\linewidth}{>{\raggedleft\scshape}p{2.2cm}X}
%\gray Period & \textbf{2010 --- 2010 (Part Time)}\\
%\gray Employer & \textbf{TecnoJr - junior enterprise} \hfill UESC - Ilhéus, Brazil\\
%\gray Position & \textbf{Software Developer (Trainee)} \hfill \textit{Volunteer}\\
%\end{tabularx}
%\end{center}

%\begin{itemize}\itemsep1.0pt
%\item Basics of software development life circle which allowed developers to get an overview of the customers needs.
%\end{itemize}

%----------------------------------------------------------------------------------------
%	EDUCATION
%----------------------------------------------------------------------------------------

\section{Formal Education/academic degrees}

\begin{center}
\begin{tabularx}{1.0\linewidth}{>{\raggedleft\scshape}p{2.2cm}X}
\gray Degree & \textbf{Bachelor in Computer Science} \hfill \textbf{July 2009 --- December 2014}\\
\gray Universities & \textbf{The State University of Santa Cruz} \hfill Ilhéus, Brazil\\ 
\gray & \textbf{Algoma University} \hfill Sault Ste Marie, Canada\\
\gray Exchange & \textbf{Science Without Borders} - \textit{1.5 year scholarship from the Brazilian Government.}
\end{tabularx}
\end{center}

\begin{itemize}\itemsep1.0pt

\item Three years as \textbf{Student Researcher} sponsored by Research Support Foundation of the State of Bahia – FAPESB.

\item \textbf{Theses in Computer Science:} Parallel Processing Using Hybrid Techniques for Applications in Transport of Particles based on Monte Carlo method. Technologies: NVIDIA CUDA, OpenMP e MPI.
   
\item \textbf{EasyKanban (2012)}. Web-based application for project management. \textit{Awarded First Prize in Computer Science Software Engineering Class Contest}.
  
\item \textbf{Application for compressing and encrypting text files} using Huffman algorithm which can reduce file's size up to 60\%. \textit{Awarded First Prize in Computer Science Programming Class Contest}, (2010). 

%\item Designed and built a Compiler that performs lexical analysis from computer programs' source code.\textit{Awarded First Prize in Computer Science Compiler Class Contest}.

%\item \textbf{Awarded First Prize in Computer Science AI class tournament} for multi-agent combat using the Gun-Tactx Platform.
  
\end{itemize}

%----------------------------------------------------------------------------------------
%	Complementary Education
%----------------------------------------------------------------------------------------

\section{Complementary Education}

\begin{center}
\begin{tabularx}{1.0\linewidth}{>{\raggedleft\scshape}p{2.2cm}X}
Nanodegree & \textbf{Udacity Deep Learning} \hfill January 2017 - Current\\
 & \textbf{Udacity Machine Learning} \hfill April 2016 - September 2016\\
Courses & \textbf{English as a Second Language (ESL)} \hfill Algoma University - 2013 - 2014\\
& \textbf{Achieve Languages - English Course} \hfill 2010 - 2013\\
Online courses & \textbf{6 computer science courses} ranging from High Performance Computing, to Deep Learning and Artificial Intelligence (certificates available upon request).\\
\end{tabularx}
\end{center}


%----------------------------------------------------------------------------------------
%	Research  Projects
%----------------------------------------------------------------------------------------
\section{Research Projects / Presentations in Events}

\begin{center}
\begin{tabularx}{1.0\linewidth}{>{\raggedleft\scshape}p{2.34cm}X}
\gray University & \textbf{The State University of Santa Cruz} \hfill Ilhéus, Brazil\\
\gray Advisor & \textbf{Esbel Tomás Valero Orellana} \hfill CV: \hyperref[Esbel Valero]{\textit{http://lattes.cnpq.br/8384020879567133}}\\
\gray Funding Institution & \textbf{Research Support Foundation of the State of Bahia – FAPESB}
\end{tabularx}
\end{center}

\subsection*{Research}

\begin{itemize}\itemsep1.0pt
\item Parallel processing on high performance stations applied to particle transport simulation using the Monte Carlo method \textit{(2012 - 2013)}.

\item Parallel processing using Graphics processing units (GPUs) applied to particle transport simulation with the Monte Carlo method \textit{(2011 - 2012)}.

\item Parallel implementations for Random Walk Algorithms - \textbf{Volunteer} \textit{(2010 - 2011)}.
\end{itemize}

\subsection*{Presentations}

\begin{itemize}\itemsep1.0pt
\item Parallel implementations of the Random Walk Algorithm

	\begin{itemize}
      \item \textit{The State University of Santa Cruz - UESC - XVIII Scientific Seminar (2012)}.
      \item \textit{The State University of Santa Cruz - UESC - Computer Week (2011)}.
	\end{itemize}
\end{itemize}

%----------------------------------------------------------------------------------------
%	Areas of Expertise
%----------------------------------------------------------------------------------------

%\section{Areas of Expertise}
%
%\begin{enumerate}\itemsep1.5pt
%\item Artificial Intelligence.
%\item Deep Learning / Machine Learning.
%\item High Parallel Computing.
%\end{enumerate}

%%----------------------------------------------------------------------------------------
%%	Productions/Presentations in Events
%%----------------------------------------------------------------------------------------
%
%\section{Productions/Presentations in Events}
%
%\begin{center}
%\begin{tabularx}{1.0\linewidth}{>{\raggedleft\scshape}p{2.34cm}X}
%\gray University & \textbf{The State University of Santa Cruz} \hfill Ilhéus, Brazil - \textbf{2011 - 2013}\\
%\gray Advisor & \textbf{Esbel Tomás Valero Orellana} \hfill CV: \hyperref[Esbel Valero]{\textit{http://lattes.cnpq.br/8384020879567133}}\\
%\gray Keywords & \textbf{Random Walk, CUDA, OpenMP, MPI}
%\end{tabularx}
%\end{center}
%
%\begin{itemize}\itemsep1.5pt
%\item \textbf{Parallel implementations of Random Walk Algorithm} \\ 
%      \textit{The State University of Santa Cruz - UESC - XVIII Scientific Seminar (2012)}.
%      
%\item \textbf{Parallel implementations of Random Walk Algorithm} \\ 
%      \textit{The State University of Santa Cruz - UESC - Computer Week (2011)}.
%\end{itemize}

%----------------------------------------------------------------------------------------
%	Participation In Events
%----------------------------------------------------------------------------------------

%\section{Participation In Events}
%
%\begin{enumerate}\itemsep1.5pt
%
%\item \textbf{XVIII UESC Scientific Seminar, 2012.} \textit{(Symposium)}
%
%\item \textbf{High Performance Computing, 2011.} \textit{(Workshop)}
%
%\item \textbf{Bahia/Alagoas/Sergipe Computing Regional School, 2011 (XI ERBASE).}
% \textit{(Congress)}
%
%\item \textbf{Regional School of High Performance Computing (I ERAD), 2011} \textit{(Congress)}
%
%\item \textbf{Robots virtualization, 2011} \textit{(Workshop)}
%
%\item \textbf{X Computer Week at the State University of Santa Cruz, 2010} \textit{(Congress)}
%
%\item \textbf{High Performance Computing, 2010} \textit{(Workshop)}
%
%\item \textbf{IX Semana de Informática da UESC, 2009} \textit{(Congress)}
%
%% \item \textbf{Gimp for creative minds, 2009} \textit{(Workshop)}
%
%\end{enumerate}

%----------------------------------------------------------------------------------------
%	EXTRA CURRICULAR ACTIVITIES
%----------------------------------------------------------------------------------------

% \section{Extracurricular Activities}
%\begin{center}
%\begin{itemize}\itemsep1.5pt
%\item Udacity - 6 computer science courses ranging from High Performance Computing, to Deep Learning and Artificial Intelligence. (certificates available upon request)
%\item Coursera - University of California, Santa Cruz , C++ for C Programmers
%\item Achieve Languages - English Course (2010 --- 2013)
%
%% \item Udacity Introduction to Computer Science Building a Search Engine
%% \item Udacity Introduction to Parallel Programming using NVIDIA CUDA
%% \item Udacity Introduction to Machine Learning.
%% \item Udacity Deep Learning
%% \item Udacity Objected Oriented JavaScript
%% \item Udacity JavaScript Design Patterns
%\end{itemize}
%\end{center}


%----------------------------------------------------------------------------------------
%	SKILLS
%----------------------------------------------------------------------------------------

\section{Skills}

\begin{tabularx}{1.0\linewidth}{>{\raggedleft\scshape}p{3.6cm}X}
\gray Languages & \textbf{C; C++; C\#; R; PHP; Python; SQL; CUDA; OpenMP; MPI; JavaScript}\\
\gray Frameworks & \textbf{AngularJS, KnockoutJS, TensorFlow, Sklearn}\\
\gray Databases & \textbf{MySQL, PostgreSQL, MongoDB}\\
\gray Technologies & \textbf{Jupyter Notebooks; Git; Visual Studio}\\
\gray Operating Systems & \textbf{Linux; MAC OS; Windows;}\\
\end{tabularx}


%----------------------------------------------------------------------------------------
%	IDIOMAS
%----------------------------------------------------------------------------------------

\section{Languages}

\begin{itemize} \itemsep1.5pt \parskip0pt \parsep0pt
  \item English and Portuguese 
\end{itemize}


%----------------------------------------------------------------------------------------
%	PROJECTS
%----------------------------------------------------------------------------------------

\section{Projects}

\begin{center}
\begin{tabularx}{1.0\linewidth}{>{\raggedleft\scshape}p{1.8cm}X}
\gray Name & \textbf{Asynchronous Actor Critic (A3C) Tensorflow implementation} \hfill Jan 2017\\
\gray Github & \href{https://github.com/thalles753/machine-learning/tree/master/projects/A3C}{Github code link}
\end{tabularx}
\end{center}

\begin{itemize}\itemsep1.5pt
\item My version of the Asynchronous Actor Critic (A3C) Tensorflow implementation from Google's DeepMind using Tensorflow and Openai Gym.
\end{itemize}

\begin{center}
\begin{tabularx}{1.0\linewidth}{>{\raggedleft\scshape}p{1.8cm}X}
\gray Name & \textbf{Street View Sequence Recognition} \hfill Sep 2016\\
\gray Github & \href{https://github.com/thalles753/machine-learning/tree/master/projects/capstone/sequence_recognition}{Github code link}
\end{tabularx}
\end{center}

\begin{itemize}\itemsep1.5pt
\item Deep Convolutional Network for recognizing sequences of digits from google maps street view images. Deployed several techniques for analyzing and synthetically increasing dataset's varieties to achieve very good results.
\end{itemize}

\begin{center}
\begin{tabularx}{1.0\linewidth}{>{\raggedleft\scshape}p{1.8cm}X}
\gray Name & \textbf{Creating Customer Segments - Unsupervised Learning} \hfill June 2016\\
\gray Github & \href{https://github.com/thalles753/machine-learning/tree/master/projects/creating_customer_segments}{Github code link}
\end{tabularx}
\end{center}

\begin{itemize}\itemsep1.5pt
\item Reviewed unstructured data to understand the patterns and natural categories that the data fits into.
Used multiple algorithms and both empirically and theoretically compared and contrasted their results.
Made predictions about the natural categories of multiple types in a dataset, then checked these
predictions against the result of unsupervised analysis.
\end{itemize}

\begin{center}
\begin{tabularx}{1.0\linewidth}{>{\raggedleft\scshape}p{1.8cm}X}
\gray Name & \textbf{Predicting Boston Housing Prices} \hfill May 2016\\
\gray Github & \href{https://github.com/thalles753/machine-learning/tree/master/projects/boston_housing}{Github code link}
\end{tabularx}
\end{center}

\begin{itemize}\itemsep1.5pt
\item Built a model to predict the value of a given house in the Boston real estate market using various statistical analysis tools. Identified the best price that a client can sell their house utilizing machine learning.
\end{itemize}

%----------------------------------------------------------------------------------------
%	REFERENCIAS
%----------------------------------------------------------------------------------------

\section{References}

\begin{itemize} \itemsep2pt \parskip0pt \parsep0pt
\item \textbf{Dr Jean-Noel Candau}, Scientist, Great Lakes Forestry Centre, Sault Ste Marie, Ontario, Canada. \textit{Jean-Noel.Candau@NRCan-RNCan.gc.ca}

\item \textbf{Dr Esbel Tomás V. Orellana}, Teacher, The State University of Santa Cruz, Ilhéus, Brazil. \textit{valero.esbel@gmail.com} 

\item \textbf{Mydiã Falcão Freitas}, Software Engineer, RCS Informática, Itabuna, Brazil. \textit{mydyfreitas@gmail.com} 
\end{itemize}

%\end{center}

\end{document}